\documentclass[12pt,letterpaper]{article}
\usepackage[utf8]{inputenc}
\usepackage{amsmath}
\setlength{\parindent}{0.0in}
\setlength{\parskip}{0.05in}
\usepackage{graphicx}
\usepackage{listings}
\usepackage{xcolor}
\usepackage{fancyhdr}
\usepackage{float}
\lstset{backgroundcolor= \color{black!5},
    language        = python,
     basicstyle      = \footnotesize,
    keywordstyle    = \color{blue},
    stringstyle     = \color{black},
    commentstyle    = \color{orange}\ttfamily,
    frame=single,
  breaklines=true,
  postbreak=\mbox{\textcolor{red}{$\hookrightarrow$}\space},
}


\begin{document}
\section{Nomor 1 Python}
Penjelasan singkat: akan dilakukan traversal BFS yang dimulai pada simpul start dan selama proses traversal
akan dilakukan pencatatan mengenai jarak tiap simpul (yang \textit{reachable} dari start position) relatif terhadap posisi start.
\vline

 Penjelasan lengkap: sebelum melakukan BFS, kita akan mengubah edge list (variable routes) menjadi adjecancy list. 

Algoritma BFS dapat dijelaskan sebagai berikut:
\begin{enumerate}
    \item buat sebuah queue $q$ yang nantinya akan memuat simpul yang akan dipreses.
    \item buat sebuah array used[] yang nantinya akan menandakan simpul mana saja yang sudah dikunjungi
    \item  mula-mula, masukan simpul $start$ pada queue $q$ dan set used[$start$]=$true$, dan untuk sembarang simpul $v$ lainnya set used[$v$]=$false$
    \item kemudian, lakukan looping sampai queue $q$ kosong. Pada setiap iterasi, pop simpul terdepan pada $q$, lakukan kunjungan pada simpul tetangga yang belum dikunjungi (used[$v$]=$false$) dan memasukan simpul tersebut pada $q$ (used[$v$] menjadi $true$)
\end{enumerate}
Sebagai hasilnya, ketika $q$ kosong, kita sudah mengunjungi setiap simpul yang \textit{reachable} dari simpul \textit{start}, dengan setiap simpul yang dikunjungi dilakukan dengan jarak terpendek.
kita hanya perlu sebuah array dist[] untuk menyimpan jarak-jarak tersebut selama proses BFS berlangsung. Penjelasan yang diminta tepat dist[end].

berikut ini adalah impelentasi dengan python:

\break 
\begin{lstlisting}
    def solve(v_list,e_list,start,end):
    """ mencari jarak terpendek dari node start ke node end
        jika diberikan vertices_list (v_list) dan edge_list (e_list).

        jika tidak ada lintasan dari start ke end, fungsi akan mengembalikan -1 sebagai hasilnya
    """

    #membuat adjecancy_list
    adj=defaultdict(list)
    for u,v in e_list:
        adj[u]+=[v]

    visited=[] # list node yang sudah dikunjungi
    queue=[(0,start)] #queue (dengan element(jarak, node)) untuk proses bfs

    #proses bfs dilakukan
    while queue:
        dist, node= queue.pop(0) #pop first element
        if node==end: #jika sudah sampai posisi end return jaraknya
            return dist
        if node in visited: #jika sudah dikunjungi lanjutkan ke proses ke node lain
            continue

        visited.append(node) # tandai nodesudah dikunjungi
        for nxt_node in adj[node]: #kunjungi setiap node tetangga
           queue.append((dist+1,nxt_node))
    return -1

\end{lstlisting}

\break

\section{Nomor 2 Python}
 Penjelasan singkat(ide): kita cari sebuah fungsi (mapping) yang memetakan sembarang papan ke \{0, 1,\dots, 63\}
 yang memiliki karaktersitik berikut: kita bisa mengubah hasil petanya ke sembarang nilai pada kodomain (\{0,1,\dots,63\}) dengan hanya dengan mengubah satu posisi coin saja.
 Jika kita bisa mencari fungsi tersebut, kita bisa meng-encode lokasi kunci pada papan bagaimanapun konfigurasinya 

 \vline

Penjelasan lengkap:

Misalkan $A$ himpunan semua konfigurasi papan yang mungkin, tinjau fungsi $f :A \rightarrow\{0,\dots,63\}$
 dengan $f(x)=  (b_ 5 b_4 b_3 b_2 b_1 b_0)_2 $ dimana
 \begin{enumerate}
    \item $ b_0=1$ jika jumlah koin bergambar pada kolom ke [1,3,5,7] adalah ganjil, $b_0=0$ jika lainnya. 
    \item $ b_1=1$ jika jumlah koin bergambar pada kolom ke [2,3,6,7] adalah ganjil, $b_1=0$ jika lainnya. 
    \item $ b_2=1$ jika jumlah koin bergambar pada kolom ke [4,5,6,7] adalah ganjil, $b_2=0$ jika lainnya. 
    \item $ b_3=1$ jika jumlah koin bergambar pada baris ke [1,3,5,7] adalah ganjil, $b_3=0$ jika lainnya. 
    \item $ b_4=1$ jika jumlah koin bergambar pada baris ke [2,3,6,7] adalah ganjil, $b_4=0$ jika lainnya. 
    \item $ b_5=1$ jika jumlah koin bergambar pada baris ke [4,5,6,7] adalah ganjil, $b_5=0$ jika lainnya. 
\end{enumerate}
\begin{table}[H]
    \centering
    \caption{pelabelan papan yang digunakan}
    \begin{tabular}{|l||l|l|l|l|l|l|l|l|}
    \hline
        r/c & 0 & 1 & 2 & 3 & 4 & 5 & 6 & 7 \\ \hline\hline
        0 & 0 & 1 & 2 & 3 & 4 & 5 & 6 & 7 \\ \hline
        1 & 8 & 9 & 10 & 11 & 12 & 13 & 14 & 15 \\ \hline
        2 & 16 & 17 & 18 & 19 & 20 & 21 & 22 & 23 \\ \hline
        3 & 24 & 25 & 26 & 27 & 28 & 29 & 30 & 31 \\ \hline
        4 & 32 & 33 & 34 & 35 & 36 & 37 & 38 & 39 \\ \hline
        5 & 40 & 41 & 42 & 43 & 44 & 45 & 46 & 47 \\ \hline
        6 & 48 & 49 & 50 & 51 & 52 & 53 & 54 & 55 \\ \hline
        7 & 56 & 57 & 58 & 59 & 60 & 61 & 62 & 63 \\ \hline
    \end{tabular}
    \label{papan}
\end{table}

mudah ditunjukkan bahwa fungsi $f$ memetakan $A$ ke himpunan $\{0,1,\dots, 63\}$ dan fungsi $f$ juga memenuhi
karaktersitik yang diinginkan karena untuk sembarang $x\in A$ dan $y\in \{0,1,\dots, 63\}$, jika $x^{*}$ adalah
papan $x$ yang koin pada lokasi $f(x) \oplus y$ dibalikkan, maka $f(x^{*})=y$
dengan fungsi $f$ tersebut, kita dapat pastikan akan selalu bisa meng-encode lokasi kuncinya,
sehingga penebak akan selalu benar menebak dengan strategi ini.

\textbf{Contoh}: Misalkan untuk sembarang papan $x$ dan kita punya $f(x)=37$ (lihat tabel \ref{papan}), namun lokasi kunci berada di $20$ (lihat tabel \ref{papan}).
Dengan mengubah posisi koin pada $37 \oplus 20 = 49 $ (lihat tabel \ref{papan}), kita punya papan $x^{*}$ sehingga $f(x^{*})=20$.

Berikut ini adalah implementasi fungsi f, infortmant, guesser-nya:

\begin{lstlisting}
    def bit_counter_strat(board):

    """program untuk melakukan strategi peng-encodingan papan,
    setiap sembarang susunan papan akan dimapping ke {0,1, ..., 63}
    """
    encode_for="" #hasil dari f(x) dalam binary
    for i in total_region:  #column activity:  mencari b0 , b1, b2
        sigma=0
        for j in i:
            for k in range (8):
                sigma+=board[k][j]
        if sigma%2==0:
            encode_for='0'+encode_for
        else:
            encode_for='1'+encode_for
    
    for i in total_region: #row activity: mencari b3, b4,b5,
        sigma=0
        for j in i:
            for k in range (8):
                sigma+=board[j][k]
        if sigma %2==0:
            encode_for='0'+encode_for
        else:
            encode_for='1'+encode_for

    
    return int(encode_for,2) #return dalam decimal

\end{lstlisting}
\begin{lstlisting}
    def informant(board, key):
    """melakukan 1  flip coin pada papan (dengan strategi) sehingga menghasilkan papan baru
    yang telah mengandung informasi mengenai lokasi kunci
    """
    num_encode=bit_counter_strat(board) #mencari f(x), x konfigurasi papan acak
    num_key= public_board[key[0]][key[1]] # mencari y dalam {0,.., 63}, ketika diberikan indeksnya
    xor_num= num_encode ^ num_key # lokasi koin yang akan dibalikan
    row, column = get_index(public_board,xor_num) #indeks koin yang akan dibalikan
    board[row][column]= 1-board[row][column] #membalikkan koin tersebut
    return board #mereturn board (x*)

\end{lstlisting}

\begin{lstlisting}
    def guess(board):

    """ melakukan decode mengenai lokasi informasi kunci pada papan yang sudah dilakukan flip
        coin
    """
    num_encode=bit_counter_strat(board) #f(x*) [mendeocode lokasi kunci pada papan dari informant]
    print(get_index(public_board, num_encode)) #mereturn indeksnya
\end{lstlisting}

\break

\end{document}